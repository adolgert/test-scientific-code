


Our attention shifts to asking, which parameter sets should I test
for this function? Which parameter sets are meaningfully different
test? How do I know when I've tested enough of them?


\section{disaggregating function}

We could be testing a function where we know what answer it should
give. For instance, take a function which disaggregates its input
vector into finer bins.
\begin{lstlisting}
function disaggregate(coarse, coarse_bins, fine_bins)
\end{lstlisting}
For the unit test, we can create a smooth-enough function on
the fine bins, aggregate it to the coarse bins, and use
that as our known input.


\section{Faults in Mathematical Code}\label{sec:faults-and-failures}

For each section, ask what faults does this technique limit?
For instance, using a limit test for constant-mortality mean age
ensures you don't have a numerical fault.

\subsection{Fault versus Failure}
Jorgensen's \emph{Software Testing: A Craftsman's Approach} distinguishes
faults from failures~\citep{jorgensen2013}.
A fault is the characters that were typed
incorrectly. A failure is observation that something went wrong.
For instance, there may be a fault that 0.3 is assigned to an integer
instead of assigning it to a double. The failure will happen later,
in the code that divides $1 / 0$.

Any failure is a problem. We don't want to throw exceptions or segfault.
Worse for mathematical code is when an error in the code isn't visible
but does give a wrong answer.
The worst problem is faults that aren't failures.


\subsection{Classes of Faults}
Jorgensen gives a set of fault categories and examples, elided here:
\begin{itemize}
    \item Input/Output - incorrect input accepted, incomplete result
    \item Logic - missing cases, test of wrong variable
    \item Computation - incorrect algorithm, parenthesis error
    \item Interface - parameter mismatch, call to wrong procedure
    \item Data - Incorrect type, wrong data, off by one
\end{itemize}
These all apply and, yes, read Jorgensen, but most of this article 
expands upon a single category of fault: incorrect algorithm.

This document covers two different kinds of failure-free faults.
\begin{itemize}
\item Simple data manipulation that doesn't do what the user
   expected it to do.
   
\item Mathematical functions that return a result that looks
   reasonable but isn't correct.
\end{itemize}
While these faults are within the taxonomy of faults
in any book on testing, we can discuss some of the particular
challenges that complicated mathematical functions introduce.